Si suggeriscono i principali miglioramenti da apportare al sistema realizzato.

Il più immediato riguarda sicuramente la trasmissione dei dati
dai sensori al programma di calcolo.
La trasmissione via cavo va sostituita con tecnologie \textbf{senza fili}
come il Bluetooth
\textbf{è il termine corretto? cercare se è proprietario e il nome ufficiale},
per non perdere il vantaggio della mobilità offerta dagli accelerometri.

Il numero di esempi dev'essere aumentato per raggiungere la \textbf{validità statistica}.
Si ritiene però opportuno mantenere un simile rapporto (25\%)
tra gli esempi provenienti dagli esperti e quelli provenienti dai non esperti.
Primo, perché tipicamente l'istruttore mostra il gesto
molte meno volte di quante lo eseguano gli allievi.
Secondo, perché in linea di principio
c'è \emph{una} maniera di eseguire il gesto correttamente
e molte di commettere errori.

Per aumentare la mole di esempi e cercare di diffondere il sistema capillarmente,
si dovrà acquisire dati da \textbf{differenti piattaforme} hardware.
Questo rende necessari l'identificazione di tali dispositivi
e la normalizzazione dei dati.

Aumentare invece il numero di sensori è necessario a fornire la diagnostica,
a stabilire cioè in quale istante o quale segmento del gesto è stato commesso l'errore
(e magari suggerire come evitarlo).
Va realizzato quindi un indumento apposito che vincoli strettamente i sensori all'arto.

A seconda del campo in cui si vuole applicare il sistema,
i dati potranno essere elaborati dal microcontrollore o dallo \textit{smartphone},
se in locale; da un server se in remoto.
