La discreta percentuale di successi ottenuta nell'esperimento
non lascia escludere di poter valutare gesti motori anche complessi
con un solo accelerometro, anche di fascia bassa.

Il classificatore realizzato si dimostra eccessivamente \emph{severo},
giudicando come $sbagliati$ anche due esempi della maestra.
Questo comportamento potrebbe tuttavia essere causato da imperfezioni
nell'esecuzione e non da un errore di classificazione.

Per le considerazioni fatte nella \textsc{Sottosezione~\ref{rif:problema}}
è comunque preferibile che il sistema dia falsi negativi --come in effetti fa,
piuttosto che falsi positivi.

I risultati sono incoraggianti 
seppure il numero di esempi non sia statisticamente significativo.