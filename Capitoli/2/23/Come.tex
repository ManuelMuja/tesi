\subsection {Protocollo di addestramento della \ac{svm}}
I creatori della libreria $libsvm$ Chang e Lin
danno alcuni suggerimenti
per iniziare presto ad avere risultati soddifacienti con le \ac{svm}.
\begin{itemize}
  \item {trasformare i dati nel formato di un pacchetto svm}
  \item {eseguire uno scalaggio semplice dei dati}
  \item {considerare i kernel a base radiale (gaussiani)}
  \item {crossvalidare per trovare i $C$ e $\gamma$ migliori}
  \item {usare questi $C$ e $\gamma$ per addestrare sull'intero insieme di adestramento}
  \item {testare}
\end{itemize}

\subsubsection {I parametri $\mathbf{C}$ e $\mathbf{\gamma}$}
Il parametro $C$ \`e il costo.
Serve a rendere pi\`u laschi i vincoli dell'ottimizzazione
\begin{equation}
  \label{eq:vincoliC}
  0 \leq \alpha_i \leq C
\end{equation}
Necessario quando le due classi \emph{non} sono separabili,
\`e utile quando le classi sono \textit{apparentemente} separabili,
cio\`e l'insieme di addestramento le mostra separabili
ma dagli insiemi di validazione o test appare che non lo sono.

In \textsc{Figura~\ref{fig:costo}} i vettori a tratto continuo
rappresentano gli esempi di addestramento.
La macchina separa le due classi senza alcuna tolleranza per eventuali \textit{outliers}.
Quando nella fase di test appaiono i vettori rappresentati col tratteggio, la prima divisione di dimostra inadeguata, in favore di un piano separatore più ovvio ma con una certa tolleranza.


Il parametro $\gamma$ fa riferimento alla \textsc{Formula~\ref{eq:rbfk}} del kernel a base radiale (\ac{rbfk})
\begin{equation}
  \label{eq:rbfk}
  k(\boldsymbol{x_1}, \boldsymbol{x_1}) = 
    \gamma\,||\boldsymbol{x_1}-\boldsymbol{x_2}||^2
\end{equation}.

ma non \`e utilizzato nell'esempio
perch\'e si usa  un kernel polinomiale di secondo grado

\begin{equation}
  \label{eq:rbfk}
  k(\boldsymbol{x_1}, \boldsymbol{x_1}) = 
    (\boldsymbol{x_1}\cdot\boldsymbol{x_2} +1)^2
\end{equation}.



\begin{figure}
\centering
\begin{tikzpicture}

% \draw [red] (-3,-3) grid (3,3);
\draw (-1,1) circle (.25);
\draw (-1,-1) circle (.25);
\draw (-2,-1) circle (.25);
\draw (-2,-2) circle (.25);
\draw (-3,1) circle (.25);
\draw (-.5,0) circle (.25);
\draw [dashed] (.5,0) circle (.25);


\draw (1,1) ++(-.2,-.2) rectangle ++(.4,.4);
\draw (1,-1) ++(-.2,-.2) rectangle ++(.4,.4);
\draw (2,-1) ++(-.2,-.2) rectangle ++(.4,.4);
\draw (2,-2) ++(-.2,-.2) rectangle ++(.4,.4);
\draw (3,1) ++(-.2,-.2) rectangle ++(.4,.4);
\draw (-.25,-1.7) ++(-.2,-.2) rectangle ++(.4,.4);
\draw [dashed] (.2,1) ++(-.2,-.2) rectangle ++(.4,.4);

% \node [fill] at (3,3) {};
\draw (-1,-3) -- (1,3)
 node[pos=.7,sloped,above]{};
 \draw [dashed] (0,-3) -- (0,3);
%  node[pos=.8,sloped]{perceptron};


\end{tikzpicture}
\caption{Importanza del parametro di costo}
\label{fig:costo}
\end{figure}
