Si propone un'esperimento con un gesto più complesso
di quelli visti nella \textsc{Sezione~\ref{sez:biomeccanica}},
il \textit{Port de bras} della danza classica.

\subsection{Descrizione del movimento} \label{ssez:descrizionedelmovimento}
I motivi per cui è stato scelto questo gesto sono  
\begin{itemize}
	\item coinvolge soltanto un arto, lasciando fermo il resto del corpo monte della spalla
    \item è abbastanza lento da poter essere rilevato con tutti gli accelerometri disponibili, sia rispetto al campo di misura, che alla frequenza di campionamento
    \item è semplice dal punto di vista cinematico
    \item è altrettanto facile da sbagliare
\end{itemize}

\paragraph{Fasi del movimento}
\begin{enumerate}
	\item \textbf{Posizione di partenza:} con il braccio rotondo e la spalla giù, portare la mano 5cm in avanti rispetto alla posizione di riposo
    \item \textbf{Elevazione:} tenendo la spalla bassa e il gomito sostenuto, alzare la mano fino all'altezza dell'ombelico
    \item \textbf{Apertura:} aprire il braccio, sempre rotondo \emph{non} al di fuori del campo visivo e \emph{non} dietro alla spalla.
\end{enumerate}

\paragraph{Ulteriori precisazioni e possibili errori}
\begin{itemize}
    \item [$\checkmark$] attenzione a mantenere il braccio ovale
    \item [$\checkmark$] alzare all'altezza dell'ombelico, non del volto
    \item [$\checkmark$] prima si sale e \emph{poi} si apre, non tutto insieme
	\item[$\times$] mano decentrata
   	\item[$\times$] gomito disteso
	\item[$\times$] alla fine, gomito troppo basso rispetto alla spalla
   	\item[$\times$] mano troppo indietro
\end{itemize}






\subsection{Fase di acquisizione e raccolta dei dati}

\paragraph{Strumenti e persone} 

\begin{itemize}
  \item accelerometro e applicazione per la raccolta dei dati a bordo di un telefono cellulare
  \item telecamera
  \item una persona esperta
  \item due persone non esperte.
\end{itemize}
Nota bene: tutti gli interessati sono destrimani.

\paragraph{Preparazione}
D'accordo con l'esperta,
si è deciso di raccogliere un numero di esempi
simile a quello che si avrebbe all'inizio di una seduta di allenamento,
con l'istruttrice che per la prima volta spiega l'esercizio agli allievi:
\begin{itemize}
	\item[10] esempi corretti della maestra
    \item[10$\times$2] esempi degli allievi da valutare (verosimilmente scorretti) 
    \item[2$\times$Err] esempi scorretti della maestra
    \item[5] altri esempi corretti della maestra
    \item[10$\times$2] esempi degli allievi da valutare (verosimilmente migliori di prima)
    \item[5$\times$2] movimenti intenzionalmente scorretti e rumorosi
\end{itemize}
A turno gli esecutori prendono lo \textit{smartphone} e
si posizionano in modo da non avere ostacoli che impediscano i movimenti
ed avere una buona luce per le riprese.
Si cerca di far partire contemporaneamente registrazione e riprese.
Gli esempi sono registrati sulla stessa traccia.
Si pone un'attesa di circa due secondi prima del gesto,
una di tre secondi alla fine, col braccio ancora alto),
poi si riabbassa il braccio e,
una volta in posizione di partenza,
si attendono di nuovo i due secondi e si ripete.



\subsubsection{Estrazione degli esempi dalle tracce}
Gli esempi raccolti nella fase precedente vengono estratti dalle registrazioni,
che ne contengono da cinque a dieci.

Come scritto nel paragrafo precedente,
si acquisiscono cinque, dieci esempi per traccia
che devono essere poi estratti.

A seconda della complessità del movimento e
della facilità con cui si possono riconoscere i \textit{pattern} nelle tracce,
l'estrazione può essere più o meno automatizzata.
Nel caso in questione, si è preferito \textbf{prendere} i \textit{timestamp}
dalle riprese video e \textbf{ritagliare} quindi i segmenti in MatLab.
Ogni esempio ha un suo tempo di esecuzione e
conseguentemente un diverso numero di campioni,
quindi si è deciso di uniformarne la durata a cinque secondi
(equivalenti a cento campioni),
e \textbf{porre gli altri campioni costanti}.

Prima di questa fase è consigliabile filtrare le tracce con un passabasso per ridurre il rumore,
poi ridurre la frequenza di campionamento (\textsc{Sottosezione~\ref{ssez:filtraggio}}).
\subsection{Fase di Elaborazione}

{\bfseries Considerazione sulla fase di elaborazione,
come ad esempio la crossvalidazione su $C$}

% \subsubsection{Risultati dell'esperimento}