\begin{center} {\bfseries Istruzioni per la versione elettronica} \end{center}
\paragraph{}
	Il documento \`e un ipertesto e contiene pertanto dei collegamenti cliccabili.
	Sono cliccabili le voci dell'indice,
	i rimandi alle note a pi\`e di pagina,
	le citazioni dei riferimenti bibliografici,
	gli elenchi di figure, tavole e codici.
	Il programma di lettura dovrebbe fornire un pulsante \textsf{Indietro},
	con cui ritornare rapidamente al punto di testo in cui si leggeva
	prima di cliccare sul collegamento.
	
	La quasi totalit\`a delle immagini sono create o importate in un formato vettoriale
	che permette di ingrandirle pi\`u die dieci volte.
	
	La numerazione delle pagine inizia dal primo capitolo.
	Le precedenti sono numerate in romano minuscolo
	da `i' (la pagina del titolo) a `v' (quella dell'indice).
	La funzione \textsf{Vai alla pagina} dovrebbe
	riconoscere correttamente questa numerazione
	ma alcuni software (anche le stampanti!) potrebbero iniziare 
	a numerare le pagine direttamente dal titolo.
																\vspace{1in}
\begin{center} {\bfseries Instructions for the electronic version} \end{center}
\paragraph{}	\selectlanguage{english}
	Table of contents, footnotes, references,
	lists of figures, tables and codes have clickable links.
	The text reader should provide a \textsf{Back} button. 
	Images are more than ten times zoomable.
	
	Pages are numbered lowercase roman in the front matter
	from `i' (title page) to `v' (table of contents).
	Function \textsf{Go to page} should correctly work with these pages too,
	but some readers (and print utilities!) could start arabic numbering from title page
	without resetting at the first chapter.