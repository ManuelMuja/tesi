%%%%%%%%%%%%%%%%%%%%%%%%%%%%%%%%%%%
\vfill \vspace{12pt}
%\vspace{0.25in} \hrule \vspace{0.1in}

\paragraph{}
Prima di procedere con l'esposizione \dots occorre affrontare alcune questioni.




%%%%%%%%%%%%%%%%%%%%%%%%%%
\vfill
\subsection{Acquisizione}

\paragraph{Dove piazzare i sensori?}
Sicuramente uno all'altezza del polso o del dorso della mano.
Potrebbero essere necessari ulteriori due accelerometri,
uno all'altezza del gomito, l'altro alla spalla.
Si preferisce comunque ridurre il numero di sensori al minimo,
possibilmente a uno (sul polso).

Sebbene sia teoricamente possibile spostare il gomito
tenendo fermi polso e spalla
(fermi rispetto a un sistema fisso, non alla catena cinematica),
un movimento del genere risulta quantomeno innaturale
ed è inverosimile che sia di qualche utilità sportiva.

Ai fini della valutazione della correttezza del gesto
si assume quindi che un errore commesso a livello della spalla
o del gomito si propaghi lungo la catena
potendo quindi essere rilavato da un sensore a valle.
Diverso sarà il discorso quando si vorrà correggere il gesto
e non basterà sapere \emph{se} è stato commesso un errore
ma occorrerà sapere almeno \emph{dove e quando}.


%%%%%%%%%%%%%%%%%%%%%%%%%%
\vfill
\subsection{Elaborazione}






%%%%%%%%%%%%%%%%%%%%%%%%%%
\vfill
\subsection{Verifica}