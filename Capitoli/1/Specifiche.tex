%%%%%%%%%%%%%%%%%%%%%%%%%%%%%%%%%%%
\vspace{12pt}
%\vspace{0.25in} \hrule \vspace{0.1in}



\subsection{Colloquio con l'esperta}
Durante un colloquio con un'esperta in materia
si è scelto di studiare il \textbf{Port de Bras},
movimento fondamentale della \emph{danza classica}.
Il gesto è stato scelto,
oltre che per le sue caratteristiche cinematiche,
perché permette di concentrare lo studio su un solo arto
pur senza perdere generalità nella soluzione.

In questa fase, l'esperta ha definito il protocollo di esecuzione
del gesto corretto e i principali errori.
Per approfondimenti, vedere la \textsc{Sottosezione~\ref{ssez:descrizionedelmovimento}}.




\subsection{Obiettivi}
Una volta che il \emph{gesto corretto} è definito,
si può cercare una soluzione al problema specifico di questo lavoro.

Si vuole un sistema
costituito da un dispositivo facilmente maneggiabile
e da un programma di calcolo,
che valuti il movimento eseguito
come lo valuterebbe un esperto.

Si realizzerà un \emph{classificatore binario},
capace di rispondere con $vero$ o $falso$
alla domanda
\begin{quote}
il movimento appena eseguito \emph{appartiene} all'insieme dei gesti corretti?
\end{quote}.




\subsection{Vincoli}
La soluzione è vincolata dall'utilizzo di un basso numero di sensori
e dalla semplicità di definizione delle regole.




\subsection{Risorse}

Le risorse a disposizione sono accelerometri triassiali per l'acquisizione,
tecniche di apprendimento automatico per la classificazione,
il programma di calcolo MatLab per l'elaborazione numerica dei dati
e il microcontrollore Arduino per l'interfaccia trai sensori e il calcolatore.
