Nei prossimi paragrafi si dà al lavoro un inquadramento generale.


%% Definizione
	\subsection{Il gesto motorio} \label{ssez:definizione}

\begin{quotation}
	``\textbf{Il movimento} è il risultato dell'attivazione
	di un limitato distretto muscolare che produce lo spostamento nello spazio
	di una o più articolazioni.
	Esempi: \emph{flessione di un dito}, rotazione di un polso.
	\textbf{L'atto motorio} è il \emph{risultato di più movimenti},
	eseguiti sinergicamente e in maniera fluida, che coinvolgono più articolazioni.'' 
%	-- Mandolesi L.(2012), \emph{Neuroscienze dell'attività motoria.} Springer, Milano
\cite{cit:mandolesi}
\end{quotation} 

\begin{quotation}
	``\textbf{Il gesto motorio} raffigura la comunicazione non verbale
	mediante l'atto visibile relativo all'esecuzione,
	più o meno abile, di una risposta motoria.
	La somma di più gesti motori in successione è denominata \textbf{atto motorio}.''
%	-- Lestini G., \emph{Il significato di psicomotricità.} http://motricitascuola.altervista.org/
\cite{cit:lestini}
\end{quotation}

La definizione è ambigua.
Vista l'astrattezza del termine \emph{gesto},
si è preferito nel titolo del lavoro specificare che si tratterà di gesti \emph{motor\^i}.
Il sinonimo \emph{movimento},
troppo generico per essere inserito nel titolo di una tesi in ingegneria,
sarà invece utilizzato nel testo
una volta stabiliti gli ambiti di studio: elettronica, neuroscienze, biomeccanica.





% Problema
	\subsection{Il problema}

Il problema fondamentale al quale si cerca di dare una soluzione
è utilizzare tecniche non invasive per fornire dati quantitativi sui movimenti
a medici e pazienti (come ad allenatori ed atleti),
che possano essere loro di aiuto a migliorare la qualità degli esercizi
(sportivi o fisioterapici)
e a prevenire infortuni.

Particolare attenzione va posta al problema dell'\emph{autonomia} dei soggetti.
Se esercitarsi porta benefici, farlo in modo errato può provocare danni alla parsona.
D'altro canto non è possibile per un esperto garantire assistenza costante,
per fornire indicazioni e correzioni, per lo meno per questioni di costo.

È per questo che ci si sforza di fornire soluzioni per il supporto domestico ai pazienti,
come anche software per la ginnastica fai-da-te.




% Stato dell'arte
	\subsection{Lo stato dell'arte}
Le soluzioni proposte per questo genere di problemi
sono solitamente basate su tecniche di visione artificiale.
Queste richiedono però l'installazione di telecamere
e una luce appropriata alle riprese,
necessità che vincolano i movimenti all'interno di una stanza.
Sono soluzioni valide in un laboratorio medico o in una palestra,
meno valide in ambiente domestico,
quasi improponibili in un campo sportivo o in un parco.



Tra le soluzioni basate sugli accelerometri
le più pertinenti da citare in questo testo
sono un guanto proposto per migliorare la tecnica
nel lancio dei dardi \footnote{\,\cite{cit:guanto}}
e una manica per prevenire gli infortuni nel baseball \footnote{\cite{cit:manica}}.





