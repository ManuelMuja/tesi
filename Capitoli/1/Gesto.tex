\paragraph{Vediamo}
Cerchiamo adesso di specializzare sul mio sistema il processo appena visto.

%% Definizione
	\subsection{Definizione}

\begin{quotation}
	\textbf{Il movimento} è il risultato dell’attivazione
	di un limitato distretto muscolare che produce lo spostamento nello spazio
	di una o pi√π articolazioni.
	Esempi: \alert{flessione di un dito}, rotazione di un polso.
	\textbf{L'atto motorio} è il \alert{risultato di più movimenti},
	eseguiti sinergicamente e in maniera fluida, che coinvolgono pi√π articolazioni. 
%	-- Mandolesi L.(2012), \emph{Neuroscienze dell'attività motoria.} Springer, Milano
\end{quotation} \cite{cit:mandolesi}

\begin{quotation}
	\textbf{Il gesto motorio} raffigura la comunicazione non verbale
	mediante l'atto visibile relativo all'esecuzione,
	pi√π o meno abile, di una risposta motoria.
	La somma di più gesti motori in successione è denominata \textbf{atto motorio}.
	-- Lestini G., \emph{Il significato di psicomotricità.} http://motricitascuola.altervista.org/
\end{quotation} \cite{cit:lestini}


	\paragraph{Obiettivi}
	Un dispositivo facilmente maneggiabile
	che unito a un sistema di calcolo
	valuti l'esecuzione di un movimento.

	\paragraph{Vincoli}
	Pochi sensori (da 1 a 3) rigidamente solidali con l'arto.
	
	\paragraph{Risorse}
	Accelerometri triassiali, reti neurali, MatLab et similia.
	
%% Concettualizzazione
	\subsection{Concettualizzazione}
	
	\subsubsection{Descrizione e suddivisione}
	Una pratica della fisioterapia consiste nel suddividere un gesto -apparentemente- semplice in sottogesti elementari.
	Durante l'esecuzione del gesto da parte del paziente,
	l'esperto nota imperfezioni in queste fasi elementari
	e basandosi appunto su quale errore viene commesso in quale fase,
	pu\`o ipotizzare una patologia e quindi un percorso di cura.
	Se l'esperto non \`e un fisioterapista ma un allenatore
	e a essere sotto esame \`e un atleta che deve imparare
	un gesto tecnico, il problema \`e molto simile.
	
	Si cerca quindi di progettare un sistema di prova,
	un "sistema esperto" che possa sostituire
	o pi\`u verosimilmente aiutare allenatori e fisioterapisti
	a individuare e correggere gli errori.
	
	\begin {itemize}
	\item [Acquisizione] Nello specifico si vorrebbe utilizzare da uno a tre accelerometri triassiali,
	\item [Montaggio] {montati su un guanto o su una manica,}
	\item [Collegamenti] collegati via cavo o radio con un processore
	\item [Elaborazione 1] che si occupi di tutta la parte di filtraggio,
	\item [Elaborazione 2] estrazione delle caratteristiche,
	\item [Elaborazione 3] classificazione (clustering)
	\item [Elaborazione 4] e valutazione.
	\end {itemize}
	
	Con l'aiuto di un esperto umano
	bisogner\`a definire cosa sia giusto e cosa sbagliato,
	stabilire quali siano alcuni criteri di distinzione
	in maniera da indirizzare la macchina verso la classificazione.
	Stabilite le caratteristiche da estrarre
	e fornite alla macchina le opportune operazioni da eseguire,
	alle caratteristiche pu\`o anche essere assegnato un peso.
	Queste possono essere propriet\`a statistiche dei segnali
	oppure una determinata forma nel tempo
	o nel dominio della frequenza.
	
	I dati cos\`i filtrati e sottodimensionati potranno quindi
	essere dati in pasto a una o pi\`u reti neurali
	che si occupino della classificazione e della valutazione.
	Questi processi sono uno non-supervisionato, l'altro supervisionato.


\begin{landscape}
\centering
\resizebox{1.5\textwidth}{!}{%
\Tree[.Valutazione
		[.Acquisizione
			[.Meccanica
				Manicotto
				{Catena cinematica}
			] % Meccanica
			[.Elettronica
				[.Sensori
					Accelerometri
				] % Sensori
				[.Strumentazione
					[.Condizionamento
						{Front-end}
						{Filtraggio hw}
					] % Condizionamento
					Trasmissione
				] % Strumentazione
			] % Elettronica
		] % Acquisizione
		[.Elaborazione
			[.Strumentazione
				{Calcolatore\\con MatLab}
				{Meglio un\\µcontrollore}
			] % Strumentazione
			[.Algoritmi
				[.{Filtraggio sw}
					Smoothing
					Picchi
				] % Filtraggio sw
				[.Classificazione
					{Estrazione\\caratteristiche}
					{Scelta\\caratteristiche}
				] % Classificazione
			] % Algoritmi
		] % Elaborazione
	]} % Valutazione
\end{landscape}


\eject
\begin{framed} \begin{verbatim}
  1   Valutazione
  2   |__ Acquisizione
  3   |   |__ Meccanica
  4   |   |   |__ Manicotto solidale col movimento
  5   |   |   |__ Studio della catena cinematica
  6   |   |_ Elettronica
  7   |      |__ Sensori
  8   |      |   |__ Accelerometri
  9   |      |__ Strumentazione
 10   |          |__ Condizionamento
 11   |          |   |__ Front-end
 12   |          |   |__ Filtraggio hw
 13   |          |__ Trasmissione -->-+
    13|                               |
  1   |__ Elaborazione <-<-<-<-<-<-<--+
  2   |   |__ Strumentazione
  3   |   |   |__ Calcolatore con MatLab
  4   |   |   |__ (Era meglio un bel µ-controllore)
  5   |   |__ Algoritmi
  6   |       |__ Filtraggio sw
  7   |       |   |__ Smoothing
  8   |       |   |__ Conta dei picchi
  9   |       |__ Classificazione
 10   |           |__ Estrazione delle caratteristiche
 11   |           |__ Scelta delle caratteristiche
    24|
  1   |__ Verifica sperimentale
  2       |__ Cosa si vuole verificare
  3       |__ Stabilire criteri di valutazione
  4       |__ Come si vuole verifcare
  5       |   |__ Esperimenti possibili
  6       |   |__ Esperimenti fattibili
  7       |   |__ Impostazione di un esperimento
  8       |__ Realizzazione di un esperimento
  9           |__ Risultati Sperimentali
    33
\end{verbatim} \end{framed}