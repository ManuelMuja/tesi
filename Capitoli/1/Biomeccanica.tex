Si ritiene opportuno introdurre qualcuno dei concetti fondamentali della biomeccanica,
per meglio comprendere le scelte fatte in fase di progettazione
\footnote{\cite{donskoi}}.

\begin{quote}
 {L'oggetto} della biomeccanica sono le azioni motorie dell'Uomo come sistema
di movimenti mutuamente coerenti e posture del corpo.
\end{quote}

\begin{quote}
 {Il metodo} della biomeccanica \`e basato su un'analisi e una sistesi sistematica
del movimento sulla base di caratteristiche quantitative,
soprattutto la modellazione matematico-cibernetica del movimento.
\end{quote}


\begin{quote}
I muscoli di norma non funzionano isolati bens\`i a gruppi.
Le interazioni avvengono all'interno dei gruppi muscolari e anche tra gruppi.
\end{quote}
Quindi può non essere necessario monitorare ogni muscolo
o segmento della catena cinematica per valutare il movimento.

\begin{quote}
La quantit\`a di gradi di libert\`a corrisponde al numero dei possibili movimenti
angolari e lineari indipendenti di un corpo
\textit {Si dice \emph {libero} un corpo che non sia costretto nei propri movimenti.}
Tale corpo pu\`o spostarsi in tre dimensioni e ruotare su tre assi
e ha quindi sei gradi di libert\`a.
Fissando uno, due, tre punti si riducono i gradi di libert\`a a tre, uno, zero.
\end{quote}


\begin{quote}
La quantit\`a di assi principali di un'articolazione corrisponde
alla quantit\`a di gradi di libert\`a di un arto rispetto ad un altro.
Il piano di movimento \`e perpendicolare all'asse di rotazione
e caratterizza l'orientamento e lo spostamento di un arto.
L'ampiezza del movimento rappresenta lo spostamento angolare
di un arto da una posizione limite a un'altra.
\end{quote}