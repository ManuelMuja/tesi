\paragraph {L'oggetto} della biomeccanica sono le azioni motorie dell'Uomo come sistema
di movimenti mutuamente coerenti e posture del corpo.

\paragraph {L'ambito} della ricerca biomeccanica sono i principi meccanici e biologici
della formazione del movimento e le particolarit\`a della sua esecuzione.
La biomeccanica pu\`o essere \emph {generale}, quando studia tutti i movimenti di tutti gli arti;
\emph {speciale}, quando studia movimenti e arti specifici degli esercizi ginnici,
di un lavoro o della riabilitazione.

\paragraph {Compito generale} dell'analisi del movimento consiste nella valutazione
dell'efficacia del dispendio energetico per raggiungere un certo scopo.

% \paragraph {Nella teoria biomeccanica} si tratta la costruzione e le caratteristiche
% cos\`i come lo sviluppo del corpo umano come sistema biomeccanico;
% l'efficacia dell'azione motoria come sistema motorio;
% la formazione e il completamento del movimento durante l'azione motoria.
% 
\paragraph {Il metodo} della biomeccanica \`e un'analisi e una sistesi sistematica
del movimento sulla base si caratteristiche quantitative,
soprattutto la modellazione matematico-cibernetica del movimento.

\paragraph {Il corpo umano come sistema biomeccanico}
Divisione tra sistema attivo (l'intero corpo, l'apparato locomotore)
e sistema passivo (organi interni, tessuti).
Di particolare interesse per l'analisi dei movimenti dell'Uomo
\`e il sistema biomeccanico del suo apparato locomotore.
Questo serve da

\begin {enumerate}
	\item Fonte di energia;
	\item Meccanismo di trasmissione delle forze;
	\item Oggetto mobile;
	\item Sistema di controllo.
\end {enumerate}

\paragraph {Catene biomeccaniche}
Apparato locomotore come catena cinematica.
Le articolazioni e i loro collegamenti sono sotto l'effetto delle forse agenti (carico).
\textit {Dicesi \emph {carico}: Le forze agenti sui corpi e la totalit\`a delle forze che provocano deformazioni.}

\par {Le deformazioni elastiche} entrano nel corpo sotto l'effetto di un carico.
Se non si presenta alcun carico, non si verificano deformazioni.

\par {I collegamenti articolari} nelle catene biomeccaniche
permettono una mooltitudine di movimenti.
Direzione e intensit\`a (forma spaziale del movimento) sono dipendenti
dal punto di collegamento e dall'efficacia dei muscoli.

\paragraph {Una coppia cinematica} rappresenta un collegamento mobile tra due articolazioni.
L'arto di collegamento porta con s\`e una limitazione al movimento relativo.
La mobilit\`a possibile inelle catene biomeccaniche permette alle articolazioni
precise possibilit\`a di movimento relativo.

Si distinguono le seguenti limitazioni
\begin {enumerate}
	\item [Geometrica] costanza della variazione di posizione in una direzione a piacere;
	\item [Cinematica] limite di velocit\`a, ad esempio attraverso i muscoli antagonisti;
\end {enumerate}

Esistono le seguenti coppie cinematiche:
\begin  {enumerate}
	\item [Progressiva] un'articolazione pu\`o scambiarsi con un'altra (???) movimento laterale della mascella;
	\item [Coppia di rotazione] Es: rotazioni nelle articolazioni cilindriche e cerniere del corpo;
	\item [Coppia a vite ???] con collegamento tra i movimenti di traslazione e rotazione (Es: ginocchio).
\end  {enumerate}
\textit {I collegamenti che permettono una rotazione della coppia articolare si dicono \emph {cerniere}.}

\paragraph {L'interazione nei gruppi muscolari}
I muscoli di norma non funzionano isolati bens\`i a gruppi.
Le interazioni avvengono all'interno dei gruppi muscolari e anche tra gruppi.

Lo sforzo di lavoro dei muscoli (lavoro dinamico) richiama i movimenti.
Lo sforzo di sostegno dei muscoli (lavoro statico) crea il sostegno necessario a quasto movimento.

\paragraph {L'azione combinata dei gruppi muscolari}
I muscoli che circondano un'articolazione si partiscono secondo il movimento in gruppi funzionali:
\begin {enumerate}
	\item [Sinergici] (muscoli con effetto comune) che esprimono lavoro di superamento;
	\item [Antagonisti] (con effetto contrario) che esprimono lavoro cedevole.
\end {enumerate}

\paragraph {L'interazione dei gruppimuscolari rispettto a resistenze diverse}
Lo sforzo dei sinergici varia secondo diverse resistenze,
corrispondentemente alla variazione di resistenza.
Gli antagonisti si contraggono in prevlenza con una resistenza che va rimpicciolendosi
(forza d'inerzia) e anche con grandi sforzi al giunto (???).

\paragraph {La ridistribuzione dello sforzo muscolare}
L'istante di attivazione e rilassamento di un muscolo rispetto al lavoro
\`e fissato attraverso la zona delal su attivit\`a e della zona ottimale,
che nel corso del movimento porta a una variazione permanente della forza muscolare,
alla ridistribuzione dello sforzo.

\paragraph {I gradi di libert\`a del movimento}
La quantit\`a di gradi di libert\`a corrisponde al numero dei possibili movimenti
angolari e lineari indipendenti di un corpo
\textit {Si dice \emph {libero} un corpo che non sia costretto nei propri movimenti.}
Tale corpo pu\`o spostarsi in tre dimensioni e ruotare su tre assi
e ha quindi sei gradi di libert\`a.
Fissando uno, due, tre punti si riducono i gradi di libert\`a a tre, uno, zero.

\paragraph {Geometria del movimento}
La quantit\`a di assi principali di un'articolazione corrisponde
alla quantit\`a di gradi di libert\`a di un arto rispetto ad un altro.
Il piano di movimento \`e perpendicolare al'asse di rotazione
e caratterizza l'orientamento e lo spostamento di un arto.
L'ampiezza del movimento rappresenta lo spostamento angolare
di un arto da una posizione limite a un altra.

\paragraph {Gli arti come leve}
Lo scheletro, che \`e costituito da ossa mobili vincolate,
rappresenta la struttura solida delle catene cinamatiche.
Gli arti della catena con le forze agenti (forsa muscolare etc)
sono trattati in biomeccanica come leve di un sistema.

\paragraph {Le leve}


\paragraph {La regola aurea della meccanica}
Quasi tutti i muscoli del corpo agiscono vicino alle articolazioni (braccio corto della leva),
cosa che porta a risparmio in caso di richieste di forza elevate.
Nelle posizioni frequenti della leva ossea le forze muscolari sono dirette sotto angoli acuti o ottusi,
cosa che porta a una perdita non trascurabile di forza muscolare (si riduce la forza di rotazione).
La forza di trazione normale permette in questo caso di fissare l'articolazione.
A seguito della particolarit\`a dell'azione della forza muscolare sulle leve ossee
\`e necessario una tensione muscolare abbastanza elevata
per l'esecuzione non solo di movimenti di forza massimale ma anche di forza rapida.
\textbf{-- mah...}

% \paragraph {L'effetto delle forze in un campo di forza variabile}
% L'Uomo esegue le proprie azioni sotto le condizioni di un campo di forza variabile.

\paragraph {La non equivalenza dell'impulso nervoso e dello spostamento}
Un determinato effetto di un movimento \`e possibile soltanto dopo
che l'impulso dal sistema nervoso centrale ha trasmesso
le condizioni di inizio del movimento,
cio\`e la tensione muscolare, der Gliederanzahl un Gliedergeschwindigkeit.

\paragraph {L'adattamento di un sistema biomeccanico}
Le qualit\`a di un sistema biomeccanico,
che lo distinguono da un sistema meccanico di corpi solidi,
aggravano il controllo motorio,
cosa che \`e superata con l'aiuto di processi di adattamento.
\textbf{????}