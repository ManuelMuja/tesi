\section{Fasi della progettazione di un sistema esperto}
\paragraph{Breve} introduzione.


\subsection{Fase iniziale del progetto della base di conoscenza}

\paragraph{Definizione della conoscenza}

\begin{itemize}
  \item {Specificare gli obiettivi}
  \item {i vincoli}
  \item {le risorse}
  \item {i partecipanti e i loro ruoli}
\end{itemize}

\paragraph{Concettualizzazione}
\begin{itemize}
  \item {descrizione dettagliata del problema}
  \item {come suddividerlo in sottoproblemi}
  \item {elementi di ognuno in termini di ipotesi, dati, concetti intermedi del ragionamento}
  \item {come queste concettualizzazioni influenzano la possibile realizzazione}
\end{itemize}

\paragraph{Rappresentazione al computer del problema}
\begin{itemize}
  \item{una scelta specifica di rappresentazione per gli elementi della concettualizzazione}
  \item{prima fase a richiedere rappresentazione al computer}
  \item{affiorano questioni di flusso e articolazione di informazioni}
\end{itemize}


\subsection{Fase di sviluppo e collaudo del prototipo}

\begin{itemize}
  \item{Una volta che si \`e scelta la rappresentazione} possiamo iniziare a implementare un sottoinsieme prototipo della conoscenza necessaria per l'intero sistema.
  \item{La scelta del sottoinsieme \`e cruciale:}d eve includere un insieme rappresentativo della conoscenza \dots tipic* del modello complessivo
  ma anche includeresottocompiti e ragionamenti che siano sufficientemente semplici da testare.
  \item{Poi} si pu\`o espandere il prototipo per comprendere pi\`u varianti e discriminazioni pi\`u raffinate.
\end{itemize}


\subsection{Raffinamento e generalizzazione della base di conoscenza}
Richiede almeno qualche mese.